\documentclass[]{article}

\usepackage[a4paper]{geometry}

\usepackage{graphicx}
\usepackage{amssymb}
\usepackage{amsmath}
\usepackage{color}
\usepackage{proof}

\newcommand{\W}[1]{\textsf{#1}}
\newcommand{\Wv}[1]{\mathbf{#1}}
\newcommand{\bs}{\backslash}
\newcommand{\arr}[2]{#1\longrightarrow #2}

\newcommand{\xrighta}{\alpha_{\diamond}^{r}}
\newcommand{\xrightc}{\sigma_{\diamond}^{r}}

\newcommand{\Xrighta}{\widehat{\alpha}_{\diamond}^{r}}
\newcommand{\Xrightc}{\widehat{\sigma}_{\diamond}^{r}}

\newcommand{\ld}{\triangleleft ^{-1}}
\newcommand{\rd}{\triangleright ^{-1}}
\newcommand{\dd}{\triangledown ^{-1}}

\newcommand{\F}[1]{\lceil #1 \rceil}
\newcommand{\Arrow}[1]{\xlongrightarrow{\displaystyle #1}}

\usepackage{tikz}
\usetikzlibrary{calc,shapes,arrows,positioning}

\newcommand{\tikzmark}[1]{\tikz[overlay,remember picture] \node (#1) {};}
\tikzset{
	string/.style = {-,semithick}
}

\newcommand{\tmark}[2]{
	\underset{\tikzmark{#2}}{\Wv{#1}}
}


\title{\textbf{Logic and Language: Exercise (Week 6)}}
\author{Orestis Melkonian [6176208], Konstantinos Kogkalidis [6230067]}
\date{}

\begin{document}
\maketitle
\section{Syntax}
\subsection{}
First, we define the rules of \textit{rightward extraction} $\Xrighta$, $\Xrightc$:
\begin{enumerate}
\begin{minipage}{0.4\textwidth}
\item[]
\[
\infer[]{\Xrighta f: (A \otimes B) \otimes \Diamond C \rightarrow D}{f: A \otimes (B \otimes \Diamond C) \rightarrow D}
\]
\end{minipage}
\begin{minipage}{0.6\textwidth}
\item[]
\[
\infer[]{\Xrightc f: (A \otimes B) \otimes \Diamond C \rightarrow D}{f: (A \otimes \Diamond C) \otimes B \rightarrow D}
\]
\end{minipage}
\end{enumerate}
We can now proceed with the derivation of
\[\mbox{$n \otimes ((n \bs n)/(s/ \Diamond \Box np)) \otimes ((np / n) \otimes n) \otimes ((np \bs s) / np)) \rightarrow n$}\] as follows:
\[
\infer[\ld]{n \otimes ((n\bs n)/(s / \Diamond \Box np) \otimes (((np/n) \otimes n) \otimes (np \bs s)/np)) \vdash n}{
	\infer[\rd]{(n\bs n)/(s / \Diamond \Box np) \otimes (((np/n) \otimes (np \bs s)/np) \vdash n \bs n}{
		\infer[/]{(n \bs n)/(s/ \Diamond \Box np) \vdash (n \bs n)/(((np \bs n) \otimes n) \otimes (np \bs s)/np)}{
			\infer[\bs]{n \bs n \vdash n \bs n}{
				\infer[1_n]{n \vdash n}{}
				&
				\infer[1_n]{n \vdash n}{}
			}
			&
			\infer[\rhd]{((np \bs n) \otimes n) \otimes ((np \bs s)/np) \vdash s/ \Diamond \Box np}{
				\infer[\Xrighta]{(((np \bs n) \otimes n) \otimes ((np \bs s)/np)) \otimes \Diamond \Box np \vdash s}{
					\infer[\ld]{((np \bs n) \otimes n) \otimes ((np \bs s)/np \otimes \Diamond \Box np) \vdash s}{
						\infer[\rd]{(np \bs s)/np \otimes \Diamond \Box np \vdash ((np \bs n) \otimes n) \bs s}{\
							\infer[/]{(np \bs s)/np \vdash (((np \bs n) \otimes n) \bs s)/ \Diamond \Box np}{
								\infer[\bs]{np \bs  s \vdash ((np \bs n) \otimes n)\bs s}{
									\infer[\rd]{(np \bs n) \otimes n \vdash np}{
										\infer[\bs]{np \bs n \vdash np \bs n}{
											\infer[1_{np}]{np \vdash np}{}
											&
											\infer[1_{n}]{n \vdash n}{}
										}
									}
									&
									\infer[1_s]{s \vdash s}{}
								}
								&
								\infer[\dd]{\Diamond \Box np \vdash np}{
									\infer[\Box]{\Box np \vdash \Box np}{
										\infer[1_{np}]{np \vdash np}{}
									}
								}
							}
						}
					}
				}
			}
		}
	}
}
\]

\section{Interpretation}
\subsection{}
\subsection{}
By working our way from the leaves of the proof tree, we get the following generalized Kronecker delta:
\begin{align*}
& \Wv{island}_{i}\otimes\Wv{that}_{j,k,l,m}\otimes\Wv{the}_{n,o}\otimes\Wv{hurricane}_{p}\otimes\Wv{destroyed}_{q,r,s}
\overset{\delta^{i,k,l,m,n,o}_{j,t,r,s,q,p}}{\xrightarrow{\hspace*{1cm}}}
\Wv{v}_r^{obj} \in \textsc{N} \\
\Wv{v}_r^{obj} = & \Wv{island}_{i}\otimes\Wv{that}_{i,j,k,l}\otimes\Wv{the}_{m,n}\otimes\Wv{hurricane}_{n}\otimes\Wv{destroyed}_{m,k,l} \quad \text{(relabeled)} 
\end{align*}

We give the matching diagram in the figure below:
\begin{equation*}
\tmark{N}{i} \quad \tmark{N}{i'} \otimes \tmark{N}{j} \otimes \tmark{S}{k} \otimes \tmark{N}{l} \quad \tmark{N}{m} \otimes \tmark{N}{n} \quad \tmark{N}{n'} \quad \tmark{N}{m'} \otimes \tmark{S}{k'} \otimes \tmark{N}{l'}
\begin{tikzpicture}[overlay,remember picture]
    \draw[string] (i.north) to [bend right=60] node[below] {i} (i'.north);
    \draw[string] (k.north) to [bend right=90] node[below] {k} (k'.north);
    \draw[string] (l.north) to [bend right=90] node[below] {l} (l'.north);
    \draw[string] (m.north) to [bend right=60] node[below] {m} (m'.north);
    \draw[string] (n.north) to [bend right=60] node[below] {n} (n'.north);
	\draw[string] (j.north) to node[right] {j} +(0, -1.6cm);
\end{tikzpicture}
\end{equation*}


\end{document}