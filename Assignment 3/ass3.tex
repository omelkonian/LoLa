\documentclass[]{article}
\usepackage[a4paper]{geometry}
\usepackage{amssymb}
\usepackage{amsmath}
\usepackage[usenames, dvipsnames]{color, xcolor}
\definecolor{MyRed}{rgb}{0.8, 0.01, 0.04}
\definecolor{MyGreen}{rgb}{0.0, 0.5, 0.0}
\colorlet{green}{MyGreen}
\colorlet{red}{MyRed}

\usepackage{stmaryrd}
\usepackage{proof}

\newcommand{\bs}{\backslash}
\newcommand{\Ra}{\rightarrow}
\newcommand{\La}{\leftarrow}
\newcommand\ol{\overline}

\newcommand{\W}[1]{\textsf{#1}}
\newcommand{\tsc}{\textsc}
\newcommand{\ceil}[1]{\lceil #1 \rceil}
\newcommand{\w}[1]{\ceil{\mathbf{#1}}}
\newcommand{\wl}[1]{\w{#1}^\ell}

\newcommand{\la}{\lambda}
\newcommand{\pair}[2]{\langle #1, #2 \rangle}
\newcommand{\gpair}[2]{\textcolor{green}{\langle} #1, #2 \textcolor{green}{\rangle}}
\newcommand{\lap}[2]{\la \pair{#1}{#2}}
\newcommand{\conj}[2]{\land \ (#1) \ (#2)}
\newcommand{\tobeta}{\to_\beta}
\newcommand{\tobetas}{\to_{\beta}^{*}}
\newsavebox\MBox
\newcommand\Cline[2]{{\sbox\MBox{$#2$}%
  \rlap{\usebox\MBox}\color{#1}\rule[-1\dp\MBox]{\wd\MBox}{0.5pt}}}
\newcommand\red[1]{\Cline{red}{#1}}
\newcommand\green[1]{\Cline{green}{#1}}

\newcommand{\otimesS}{\cdot\otimes\cdot}
\newcommand{\focus}[1]{\fbox{$#1$}}
\newcommand{\slashS}{\cdot\slash\cdot}
\newcommand{\bsS}{\cdot\bs\cdot}
\newcommand{\oplusS}{\cdot\oplus\cdot}
\newcommand{\oslashS}{\cdot\oslash\cdot}
\newcommand{\obslashS}{\cdot\obslash\cdot}

\newcommand{\note}[1]{\{\text{\textit{#1}}\}}
\usepackage{tikz}
\usetikzlibrary{calc,shapes,arrows,positioning}

\newcommand{\tikzmark}[1]{\tikz[overlay,remember picture] \node (#1) {};}
\tikzset{
	string/.style = {-,thick,shorten >=3pt,shorten <=3pt,transform canvas={yshift=-.5mm,xshift=.5mm}}
}

\title{\textbf{Logic and Language: Exercise (Week 5)}}
\author{Orestis Melkonian [6176208], Konstantinos Kogkalidis [6230067]}
\date{}
\begin{document}
\maketitle
\section{LG: continuation semantics}
\subsection{}
For each term, we compute the ILL type by first observing the polarity of the sequent and then using the table to interpret complex types.
\begin{align*}
\w{some}
=& \ceil{np/n}^\bot &\note{Negative Hypothesis} \\
=& (\ceil{np}^\bot \otimes \ceil{n})^\bot &\note{$\ceil{A/B}$ with A and B positive} \\
=& (np^\bot \otimes n)^\bot &\note{np and n positive} \\
\\
\w{popular}
=& \ceil{n/n}^\bot &\note{Negative Hypothesis}\\
=& (\ceil{n}^\bot \otimes \ceil{n})^\bot &\note{$\ceil{A/B}$ with A and B positive}\\
=& (n^\bot \otimes n)^\bot &\note{n positive}\\
\\
\w{saint}
=& \ceil{n} &\note{Positive Hypothesis}\\
=& n &\note{n positive}\\
\\
\w{arrived}
=& \ceil{np \bs s}^\bot &\note{Negative Hypothesis} \\
=& (\ceil{np} \otimes \ceil{s})^\bot &\note{$\ceil{B \bs A}$ with B positive and A negative}\\
=& (np \otimes s^\bot)^\bot &\note{np positive, s negative}\\
\\
\alpha
=& \ceil{s} \\
=& s^\bot &\note{s negative}\\
\\
z
=& \ceil{np} \\
=& np &\note{np positive}\\
\\
y
=& \ceil{n}\\
=& n &\note{n positive}\\
\end{align*}
The ILL types then are:
\[\begin{array}{cc}
\tsc{term} & \tsc{type}\\
\hline
\w{some} & (np^\bot \otimes n)^\bot \\
\w{popular} & (n^\bot \otimes n)^\bot \\
\w{saint} & n \\
\w{arrived} &  (np \otimes s^\bot)^\bot \\
\alpha & s^\bot \\
z & np \\
y & n \\
\end{array}\]
\subsection{}
\[\begin{array}{ccc}
\textsc{source type} & \textsc{constant} & \ceil{.}^\ell\\
\hline 
n/n & \textsf{popular} & \lambda\langle c,y \rangle. (c\ (\lambda z.\ \wedge (y\ z)\ (\tsc{popular}\ z)))\\
\end{array}\]
\subsection{}
\begin{enumerate}
\item 
\[
\infer[\rightharpoondown]{(np/n) \otimesS ((n/n) \otimesS n)) \otimesS (np \bs s) \vdash \focus{\lambda \alpha_0 . ( \ceil{arrived} \langle \lambda \beta_0 . \ceil{some} \langle \beta_0 , \lambda \gamma_0 . ( \ceil{popular} \langle \gamma_0 , \lambda \alpha_1 . ( \ceil{saint} \ \alpha_1 ) \rangle ) \rangle ) , \alpha_0 \rangle ) : s}}{
	\infer[\leftharpoonup]{(\ceil{arrived} \langle \beta_0. (\ceil{some} \langle \beta_0, \lambda \gamma_0.(\ceil{popular} \langle \gamma_0 , \lambda \alpha_1 .(\ceil{saint} \alpha_1) \rangle ) \rangle ), \alpha_0 \rangle ) : np \bs s \vdash \alpha_0 : ((np/n) \otimesS ((n/n) \otimes n)) \bsS  s}{
		\infer[\bs L]{\focus{(\ceil{arrived} \langle \beta_0. (\ceil{some} \langle \beta_0, \lambda \gamma_0.(\ceil{popular} \langle \gamma_0 , \lambda \alpha_1 .(\ceil{saint} \alpha_1) \rangle ) \rangle ), \alpha_0 \rangle ) : np \bs s \vdash \alpha_0 : ((np/n) \otimesS ((n/n) \otimesS n)) \bsS s}}{
			\infer[\rightharpoondown]{(np/n) \otimesS ((n/n) \otimesS n) \vdash \focus{\lambda \beta_0 . ( \ceil{some} \langle \beta_0 , \lambda \gamma_0 . ( \ceil{popular} \langle \gamma_0 , \lambda \alpha_1 (\ceil{saint} \ \alpha_1) \rangle ) \rangle ) : np}}{
				\infer[\leftharpoonup]{(\ceil{some} \langle \beta_0 , \lambda \gamma_0 . (\ceil{popular} \langle \gamma_0 , \lambda \alpha_1 . (\ceil{saint} \ \alpha_1 ) \rangle ) \rangle ) : np/n \vdash \beta_0 : np \slashS ((n/n) \otimesS n)}{
					\infer[/ L]{\focus {\langle \beta_0 , \lambda \gamma_0 . (\ceil{popular} \langle \gamma_0 , \lambda \alpha_1 . ( \ceil{saint} \ \alpha_1 ) \rangle ) \rangle ) : np/n }}{
						\infer[CoAx]{\focus{ \beta_0 : np} \vdash np}{\beta_0}
						&
						\infer[\rightharpoondown]{(n/n) \otimesS n \vdash \focus{\lambda \gamma_0 . ( \ceil{popular} \langle \gamma_0 , \lambda \alpha_1 . ( \ceil{saint} \ \alpha_1 ) \rangle ) : n}}{
							\infer[\rightharpoonup]{(\ceil{popular} \langle \gamma_0 , \lambda \alpha_1 . ( \ceil{saint} \ \alpha_1) \rangle ) : n/n \vdash \gamma_0 : n \slashS n}{
								\infer[/ L]{\focus{\langle \gamma_0 , \lambda \alpha_1 . ( \ceil{saint} \ \alpha_1) \rangle : n/n} \vdash \gamma_0 : n \slashS n}{
									\infer[CoAx]{\focus{\gamma_0 : n} \vdash n}{\gamma_0}
									&
									\infer[\rightharpoondown]{n \vdash \focus{\lambda \alpha_1 . ( \ceil{saint} \ \alpha_1 ) : n}}{
										\infer[\leftharpoonup]{\ceil{saint} \ \alpha_1 : n \vdash \alpha_1 : n}{
											\infer[CoAx]{\focus{\alpha_1 : n} \vdash n}{ \alpha_1}
										}
									}								
								}
							}
						}
					}
				}
			}
			&
			\infer[CoAx]{\focus{\alpha_0 : s \vdash s}}{\alpha_0}
		}
	}
}
\]
\item We compute the interpretation below:
\[ \ceil{\ddag} = \la a_0.(
	\w{arrived} \ \pair
		{\la \beta_0.(
			\w{some} \ \pair
				{\beta_0}
				{\la \gamma_0.(
					\w{popular} \ \pair
						{\gamma_0}
						{\la a_1.(\w{saint} \ a_1)}
				)}
		)}
		{a_0}
	)
\]
\item The adjucted $\cdot^\ell$ translations are the following:
\begin{align*}
\wl{some} &= \lap{x}{k}.(\exists\la z.(\conj{k \ \la\theta.(\theta \ z)}{x \ z})) \\
\wl{popular} &= \lap{c}{k}.(c \ (\la z.(\conj{\tsc{POPULAR} \ z}{k \ \la\theta.(\theta \ z)}))) \\
\wl{saint} &= \la c.(c \ \tsc{SAINT}) \\
\wl{arrived} &= \lap{k}{c}.(k \ \la z.(c \ (\tsc{ARRIVED} \ z))) \\
\end{align*}
We can now corroborate the $\alpha$-equivalence of the two $\cdot^\ell$ translations:
\begin{align*}
\ceil{\ddag}^\ell
&= \la a_0.(
	\red{\lap{k}{c}.(k \ \la z.(c \ (\tsc{ARRIVED} \ z)))} \\&\quad\quad \gpair
		{\la \beta_0.(
			 \lap{x}{k}.(\exists\la z.(\conj{k \ \la\theta.(\theta \ z)}{x \ z})) \ \pair
				{\beta_0}
				{\la \gamma_0.( \\&\quad\quad\quad
					\lap{c}{k}.(c \ (\la z.(\conj{\tsc{POP} \ z}{k \ \la\theta.(\theta \ z)}))) \ \pair
						{\gamma_0}
						{\la a_1.(
							\la c.(c \ \tsc{SAINT}) \ a_1
						)}
				)}
		)}
		{a_0}
	)
\\
&\tobetas \la a_0.(
	\red{\lap{x}{k}.(\exists\la z.(\conj{k \ \la\theta.(\theta \ z)}{x \ z}))} \ \gpair
				{\la z.(a_0 \ (\tsc{ARRIVED} \ z))}
				{\la \gamma_0.( \\&\quad\quad\quad
					\lap{c}{k}.(c \ (\la z.(\conj{\tsc{POPULAR} \ z}{k \ \la\theta.(\theta \ z)}))) \ \pair
						{\gamma_0}
						{\la a_1.(
							\la c.(c \ \tsc{SAINT}) \ a_1
						)}
				)}
)
\\
&\tobetas \la a_0.(\exists\la z.(\conj{
	\red{\lap{c}{k}.(c \ (\la z.(\conj{\tsc{POPULAR} \ z}{k \ \la\theta.(\theta \ z)})))}\\&\quad\quad\quad \gpair
		{\la\theta.(\theta \ z)}
		{\la a_1.(\la c.(c \ \tsc{SAINT}) \ a_1)}
	}
	{a_0 \ (\tsc{ARRIVED} \ z)})
\\
&\tobetas \la a_0.(\exists\la z.(\conj
	{
	\red{\la\theta.(\theta \ z)} \ \textcolor{green}{(}\la z.(\conj{\tsc{POPULAR} \ z}{\red{\la c.(c \ \tsc{SAINT})} \ \green{(\la\theta.(\theta \ z))}})\textcolor{green}{)}
	}
	{a_0 \ (\tsc{ARRIVED} \ z)})
\\
&\tobetas \la \textcolor{red}{a_0}.(\exists\la z.(\conj
	{\conj{\tsc{POPULAR} \ z}{\tsc{SAINT} \ z}}
	{\textcolor{red}{a_0} \ (\tsc{ARRIVED} \ z)})
\\
&\to_\alpha \la c.(\exists\la \textcolor{red}{z}.(\conj
	{\conj{\tsc{POPULAR} \ \textcolor{red}{z}}{\tsc{SAINT} \ \textcolor{red}{z}}}
	{c \ (\tsc{ARRIVED} \ \textcolor{red}{z})})
\\
&\to_\alpha \la c.(\exists\la x.(\conj
	{\conj{\tsc{POPULAR} \ x}{\tsc{SAINT} \ x}}
	{c \ (\tsc{ARRIVED} \ x)})
\\
&= \ceil{\dag}^\ell
\end{align*}
\end{enumerate}

\section{Pregroups}
\subsection{}
\begin{enumerate}
\item[(4)]
\begin{minipage}{0.5\textwidth}
\center{($\Ra$)}
\[
\infer[\Ra ]
	{ 1^l \Ra 1}
	{\infer[(1)]
		{1^l \Ra 1^l1}
		{}
	&\infer[(2)]
		{1^l1 \Ra 1}
		{}
	}
\]
\end{minipage}
\begin{minipage}{0.5\textwidth}
\center{($\La$)}
\[
\infer[\Ra]
	{1 \Ra 1^l}
	{\infer[(2)]
		{1 \Ra 1^l1}
		{}
	&\infer[(1)]
		{1^l1 \Ra 1^l}
		{}
	}
\]
\end{minipage}
\item[(5)]
($\Ra$):
\[
\infer[\Ra]{A^{rl} \Ra A}{
\infer[\Ra]{A^{rl} \Ra 1A} {
	\infer[(1)]{A^{rl} \Ra (A^{rl}A^r)A}{
		\infer[\Ra]{A^{rl} \Ra A^{rl}(A^rA)}{
			\infer[(1)]{A^{rl} \Ra A^{rl}1}{}
			&
			\infer[(2)]{1 \Ra A^rA}{}
		}	
	}
	&
	\infer[(2)]{A^{rl}A^r \Ra 1}{}
}
&\infer[(1)]{1A \Ra A}{}}
\]
($\La$):
\[
\infer[\Ra]{A \Ra A^{rl}}{
	\infer[\Ra]{A \Ra 1A^{rl}}{
		\infer[(1)]{A \Ra (AA^r)A^{rl}}{
			\infer[\Ra]{A \Ra A(A^rA^{rl})}{
				\infer[(1)]{A \Ra A1}{}
				&
				\infer[(2)]{1 \Ra A^rA^{rl}}{}
			}
		}
		&
		\infer[(3)]{AA^r \Ra 1}{}
	}
	&
	\infer[(1)]{1A^{rl} \Ra A^{rl}}{}
}
\]
\item[(6)]
($\Ra$):
\[
\infer[(5)]{(AB)^l \Ra B^lA^l}{
	\infer[\Ra]{(AB)^l \Ra ((B^lA^l)^r)^l}{
		\infer[todo]{(AB)^l \Ra ((B^lA^l)^r1)^l}{
			\infer[\Ra]{(AB)^l \Ra ((B^lA^l)^r(B^lB))^l}{
				\infer[\Ra]{(AB)^l \Ra ((B^lA^l)^r(B^l(1B)))^l}{
					\infer[(1)]{(AB)^l \Ra ((B^lA^l)^r(B^l((A^lA)B)))}{
						\infer[(1)]{(AB)^l \Ra ((B^lA^l)^r(B^l(A^l(AB))))}{
							\infer[(1)]{(AB)^l \Ra ((B^lA^l)^r((B^lA^l)(AB)))^l}{
								\infer[\Ra]{(AB)^l \Ra (((B^lA^l)^r(B^lA^l))(AB))^l}{\
									\infer[\Ra]{(AB)^l \Ra (1(AB))^l}{}
									&
									\infer[(3)]{1 \Ra (B^lA^l)^r(B^lA^l)}{}
								}
							}
						}
					}
					&
					\infer[(2)]{A^lA \Ra 1}{}
				}
				&
				\infer[(1)]{1B \Ra B}{}
			}
			&
			\infer[(2)]{B^lB \Ra 1}{}
		}
		&
		\infer[(1)]{(B^lA^l)^r1 \Ra (B^lA^l)^r}{}
	}
}
\]
\end{enumerate}

\subsection{}
We first calculate the pregroup translation of the given sequent:
\begin{align*}
& \ol{(p/((q/q)/r))/r}\\
=& \ol{(p/((q/q)/r))} \ r^\ell &\note{$\ol{A/B} = \ol{A}(\ol{B})^\ell$}\\
=& p \ \ol{((q/q)/r)}^\ell \ r^\ell &\note{$\ol{A/B} = \ol{A}(\ol{B})^\ell$}\\
=& p \ (\ol{(q/q)} \ r^\ell)^\ell \ r^\ell &\note{$\ol{A/B} = \ol{A}(\ol{B})^\ell$}\\
=& p \ ((q \ q^\ell) \ r^\ell)^\ell \ r^\ell &\note{$\ol{A/B} = \ol{A}(\ol{B})^\ell$}\\
=& p \ (q \ (q^\ell \ r^\ell))^\ell \ r^\ell &\note{rule (1) from 2.1}& \\
=& p \ (q^\ell \ r^\ell)^\ell \ q^\ell \ r^\ell &\note{rule (6) from 2.1} \\
=& p \ r^{\ell^\ell} \ q^{\ell^\ell} \ q^\ell \ r^\ell &\note{rule (6) from 2.1} \\
\end{align*}
Finally, we prove the sequent by drawing a string diagram:
\begin{equation*}
 \tikzmark{p}p \ \tikzmark{r}r^{\ell^\ell} \ \tikzmark{q}q^{\ell^\ell} \ \tikzmark{q'}q^\ell \ \tikzmark{r'}r^\ell
 \begin{tikzpicture}[overlay,remember picture]
    \draw[string] (q.center) to [bend right=90] (q'.center);
    \draw[string] (r.center) to [bend right=60] (r'.center);
	\draw[string] (p.center) to +(0, -.8cm);
  \end{tikzpicture}
\end{equation*}

\end{document}