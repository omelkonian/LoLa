\documentclass[]{article}
\usepackage[a4paper]{geometry}
\usepackage{amssymb}
\usepackage{amsmath}
\usepackage[usenames, dvipsnames]{color, xcolor}
\definecolor{MyRed}{rgb}{0.8, 0.01, 0.04}
\definecolor{MyGreen}{rgb}{0.0, 0.5, 0.0}
\colorlet{green}{MyGreen}
\colorlet{red}{MyRed}
\usepackage{tikz-qtree}

\usepackage{stmaryrd}
\usepackage{proof}

\newcommand{\LG}{\textbf{LG}}

\newcommand{\bs}{\backslash}
\newcommand{\Ra}{\rightarrow}

\newcommand{\arr}[2]{#1 \longrightarrow #2}
\newcommand{\arrr}[2]{#2 \longrightarrow #1}

\newcommand{\arrow}[3]{#1:#2\longrightarrow #3}
\newcommand{\Arrow}[3]{#2\stackrel{#1}{\longrightarrow} #3}

\newcommand{\overa}[1]{\mathop{\triangleright} #1}
\newcommand{\undera}[1]{\mathop{\triangleleft} #1}
\newcommand{\overai}[1]{\mathop{\triangleright}^{-1} #1}
\newcommand{\underai}[1]{\mathop{\triangleleft}^{-1} #1}

\newcommand{\bovera}[1]{\mathop{\blacktriangleright} #1}
\newcommand{\bundera}[1]{\mathop{\blacktriangleleft} #1}
\newcommand{\boverai}[1]{\mathop{\blacktriangleright}^{-1} #1}
\newcommand{\bunderai}[1]{\mathop{\blacktriangleleft}^{-1} #1}

\newcommand{\gmal}{\mathop{\textbf{d}}}
\newcommand{\gmar}{\mathop{\textbf{b}}}
\newcommand{\gmcl}{\mathop{\textbf{q}}}
\newcommand{\gmcr}{\mathop{\textbf{p}}}

\newcommand{\Gmal}{\mathop{\textbf{\d{d}}}}
\newcommand{\Gmar}{\mathop{\textbf{\d{b}}}}
\newcommand{\Gmcl}{\mathop{\textbf{\.{q}}}}
\newcommand{\Gmcr}{\mathop{\textbf{\.{p}}}}

\newcommand{\cut}[1]{}

\newcommand{\W}[1]{\textsf{#1}}

\newcommand{\focus}[1]{\fbox{$#1$}}

\newcommand{\nd}[2]{#1 \vdash #2}
\newcommand{\nD}[2]{#1 \vdash \focus{#2}}
\newcommand{\Nd}[2]{\focus{#1} \vdash #2}

\newcommand{\ndO}[3]{#1 \stackrel{#3} \vdash  #2}
\newcommand{\nDO}[3]{#1 \stackrel{#3} \vdash  \focus{#2}}
\newcommand{\NdO}[3]{\focus{#1} \stackrel{#3} \vdash  #2}

\newcommand{\ndU}[3]{#1 \underset{#3} \vdash  #2}
\newcommand{\nDU}[3]{#1 \underset{#3} \vdash  \focus{#2}}
\newcommand{\NdU}[3]{\focus{#1} \underset{#3} \vdash  #2}

\newcommand{\seq}[2]{#1 \vdash #2}

\newcommand{\otimesS}{\cdot\otimes\cdot}
\newcommand{\slashS}{\cdot\slash\cdot}
\newcommand{\bsS}{\cdot\bs\cdot}
\newcommand{\oplusS}{\cdot\oplus\cdot}
\newcommand{\oslashS}{\cdot\oslash\cdot}
\newcommand{\obslashS}{\cdot\obslash\cdot}
\newcommand{\Struct}[1]{\mathop{\cdot}#1\mathop{\cdot}}

\newcommand{\comu}{\widetilde{\mu}}

\newcommand{\Zip}[1]{\langle #1 \rangle}
\newcommand{\cmdL}[2]{\langle #1 \upharpoonleft #2 \rangle}
\newcommand{\cmdR}[2]{\langle #1 \upharpoonright #2 \rangle}

\newcommand{\arroww}{\rule[-5pt]{0pt}{15pt}\longrightarrow}

\newcommand{\CPS}[1]{\lceil#1\rceil}

\newcommand{\CPSlex}[1]{#1^{\ell}}

\newcommand{\Coco}[1]{\widetilde{#1}}
\newcommand{\lolli}{\multimap}


\title{\textbf{Logic and Language: Exercise (Week 3)}}
\author{Orestis Melkonian [6176208], Konstantinos Kogkalidis [6230067]}
\date{}

\begin{document}
\maketitle

\section{LGa}
\subsection{}

\begin{enumerate}
  \item Unfolding the combinator $f/g$, we get the following derivation:
  \[\infer[]
  	{\arrow{(\overa(f\circ(\overai 1_{A/B}))) \circ (\overa\underai((\undera\overai 1_{A/B'})\circ g))}{A/B'}{A'/B}}
  	{\infer
		{\arrow{\overa(f\circ(\overai 1_{A/B}))}{A/B}{A'/B}}
		{\infer
			{\arrow{f\circ(\overai 1_{A/B})}{A/B \otimes B}{A'}}
			{\infer
				{\arrow{\overai 1_{A/B}}{A/B \otimes B}{A}}
				{\arrow{1_{A/B}}{A/B}{A/B}}
			& \arrow{f}{A}{A'}}
		}
  	&\infer
		{\arrow{\overa\underai((\undera\overai 1_{A/B'})\circ g)}{A/B'}{A/B}}
  		{\infer
  			{\arrow{\underai((\undera\overai 1_{A/B'})\circ g)}{(A/B') \otimes B}{A}}
  			{\infer
  				{\arrow{(\undera\overai 1_{A/B'})\circ g}{B}{(A/B') \bs A}}
  				{\arrow{g}{B}{B'}
  				&\infer
  					{\arrow{\undera\overai 1_{A/B'}}{B'}{(A/B') \bs A}}
  					{\infer
  						{\arrow{\overai 1_{A/B'}}{(A/B') \otimes B'}{A}}
  						{\arrow{1_{A/B'}}{A/B}{A/B'}}
  					}
  				}
  			}
  		}
  	}
  \]
  \item Applying the arrow reversal transformation $(\cdot)^\dag$ to $f/g$, we get the reverse combinator:
  \begin{align*}
  (f/g)^\dag &= ((\overa(f\circ(\overai 1_{A/B}))) \circ (\overa\underai((\undera\overai 1_{A/B'})\circ g)))^\dag \\
  &= (\overa\underai((\undera\overai 1_{A/B'})\circ g))^\dag \circ (\overa(f\circ(\overai 1_{A/B})))^\dag \\
  &= (\bundera(\underai((\undera\overai 1_{A/B'})\circ g))^\dag) \circ (\bundera(f \circ (\overai 1_{A/B}))^\dag) \\
  &= (\bundera\boverai((\undera\overai 1_{A/B'})\circ g)^\dag) \circ (\bundera((\overai 1_{A/B})^\dag \circ f^\dag)) \\
  &= (\bundera\boverai(g^\dag \circ (\undera\overai 1_{A/B'})^\dag)^\dag) \circ (\bundera((\bunderai 1_{(A/B)^\dag}) \circ f^\dag)) \\
  &= (\bundera\boverai(g^\dag \circ (\bovera(\overai 1_{A/B'})^\dag))) \circ (\bundera((\bunderai  1_{B \obslash A}) \circ f^\dag)) \\
  &= (\bundera\boverai(g^\dag \circ (\bovera\bunderai 1_{(A/B')^\dag}))) \circ (\bundera((\bunderai  1_{B \obslash A}) \circ f^\dag)) \\
  &= (\bundera\boverai(g^\dag \circ (\bovera\bunderai 1_{B' \obslash A}))) \circ (\bundera((\bunderai  1_{B \obslash A}) \circ f^\dag)) \\
  \end{align*}
  Unfolding the combinator $f/g$, we get the following derivation:
  \[\infer[]
  	{\arrow{(\bundera\boverai(g^\dag \circ (\bovera\bunderai 1_{B' \obslash A}))) \circ (\bundera((\bunderai  1_{B \obslash A}) \circ f^\dag))}{B \obslash A'}{B' \obslash A}}
  	{\infer
		{\arrow{\bundera\boverai(g^\dag \circ (\bovera\bunderai 1_{B' \obslash A}))}{B \obslash A}{B' \obslash A}}
		{\infer
			{\arrow{\boverai(g^\dag \circ (\bovera\bunderai 1_{B' \obslash A}))}{A}{B \oplus (B' \obslash A)}}
			{\infer
				{\arrow{g^\dag \circ (\bovera\bunderai 1_{B' \obslash A})}{A \oslash (B' \obslash A)}{B}}
				{\infer
					{\arrow{\bovera\bunderai 1_{B' \obslash A}}{A \oslash (B' \obslash A)}{B'}}
					{\infer
						{\arrow{\bunderai 1_{B' \obslash A}}{A}{B' \oplus (B' \obslash A)}}
						{\arrow{1_{B' \obslash A}}{B' \obslash A}{B' \obslash A}}
					}
				&\arrow{g^\dag}{B'}{B}
				}
			}
		}
  	&\infer
		{\arrow{\bundera((\bunderai  1_{B \obslash A}) \circ f^\dag)}{B \obslash A'}{B \obslash A}}
  		{\infer
  			{\arrow{(\bunderai  1_{B \obslash A}) \circ f^\dag}{A'}{B \oplus (B \obslash A)}}
  			{{\arrow{f^\dag}{A'}{A}}
  			&\infer
  				{\arrow{\bunderai  1_{B \obslash A}}{A}{B \oplus (B \obslash A)}}
  				{\arrow{1_{B \obslash A}}{B \obslash A}{B \obslash A}}
  			}
  		}
  	}
  \]

\end{enumerate}

\subsection{}
\begin{enumerate}
  \item $(a\oplus b)/c \arroww a/(c\oslash b)$
  \item $(b\oplus c)\otimes a \arroww b\oplus (c\otimes a)$
\end{enumerate}

\end{document}
